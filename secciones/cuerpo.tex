\section*{\underline{TÍTULO DEL PROYECTO}}
Diseño, desarrollo e implementación de un sistema de gestión, seguimiento y obtención de indicadores relacionados a programas de formación.


\section*{\underline{OBJETIVO DEL PROYECTO}}
Objetivo general:\\

Contribuir, posibilitar y facilitar el seguimiento de los programas de formación e inclusión tecnológica para mujeres adolescentes mediante el desarrollo e implementación de un sistema de seguimiento, control y obtención de métricas para estos programas.\\

Objetivos específicos:
\begin{itemize}
	\item Identificar y permitir la visualización de indicadores relevantes y representativos que ayuden al seguimiento de los programas.
	\item Proporcionar un sistema de notificaciones que comunique a los distintos tipos de usuarios sobre eventos que puedan ser previamente configurados por ellos.
	\item Permitir la configuración de distintos perfiles con diferentes niveles de  privilegios para utilizar el sistema.\\
\end{itemize}


\section*{\underline{DESTINATARIOS}}
El Trabajo Final de Grado se realizará en el marco del proyecto UNDEX 775/2019 aprobado al Depto. de Computación e Informática de la Facultad de Ingeniería del CRUC IUA (Resolución 441/2019), en la convocatoria 2019 de UNDEF \textbf{\cite{ResolucionUndex}}. En dicho proyecto se desarrollará la estrategia a implementar mediante un sistema software para gestionar el seguimiento de los programas de capacitación tecnológica para favorecer la inclusión de la mujer. \\


\section*{\underline{BENEFICIOS ESPERADOS}}
\begin{itemize}
	\item Obtención de indicadores y métricas para toma de decisiones.
	\item Estandarización, limpieza y centralización del almacenamiento de la información de los cursos evitando pérdidas de la misma.
	\item Automatización de la carga de datos correspondiente a los cursos, lo que conlleva a un menor esfuerzo por parte de las personas encargadas de realizar estas tareas.
	\item Seguimientos de las alumnas participantes y de los programas.
	\item Facilidad de acceso a la información.
\end{itemize}


\section*{\underline{ESTUDIO TÉCNICO}}
\begin{itemize}
	\item Para el desarrollo del software se utilizaran 3 lenguajes de programación diferentes:
	\begin{itemize}
		\item Para el Backend se utilizaran los lenguajes Golang y Python.
		\item Para el Frontend se utilizara Javascript apoyándose en un framework llamado React también basado en Javascript.
	\end{itemize}
	\item Se utilizarán tanto Bases de Datos Relacionales como No Relacionales para el guardado de datos.
	\item Para desplegar el sistema se utilizará un servidor provisto por el CRUC IUA.\\
\end{itemize}



\section*{\underline{FORMULACIÓN Y VALORACIÓN DE ALTERNATIVAS}}
Para la detección de alternativas se basó en varios puntos importantes: Costo, almacenamiento de datos, posibilidad de relizar metricas, tableros y gráficos y conectividad a internet, entre otras.\\

\begin{figure}[H]
	\centering
	\includegraphics[width=\textwidth]{imagenes/alternativas.png} 
	\caption{Alternativas}
\end{figure}

Además de las alternativas presentadas en el grafico anterior, existen otras aplicaciones web similares como chart.io, klipfolio.com, qlik.com, entre otras pero todas ellas son pagas. \\


Se puede concluir entonces, que no se han detectado alternativas gratuitas y/o herramientas que favorezcan y ayuden al seguimiento de los programas de inclusión.\\


\section*{\underline{METODOLOGÍA}}
Para comenzar se procederá al estudio general de los distintos programas de formación tecnológica para la inclusión de la mujer y se hará un relevamiento de la información disponible de los cursos anteriores de este programa para poder identificar qué datos serían de interés a la hora de ingresarlos al sistema y qué información debería proveer el mismo.\\

Luego se seguirá con un análisis de las funcionalidades del sistema para poder hacer el diseño del mismo.\\

Una vez que el diseño esté finalizado, se comenzará la implementación del sistema.\\

Paralelamente, se desarrollarán y ejecutarán pruebas para asegurar el correcto funcionamiento del mismo.\\

Finalmente, se procederá a realizar pruebas del sistema completo intentando imitar lo mejor posible al funcionamiento real del mismo.\\

Para llevar a cabo todas estas actividades se utilizará una metodología Ágil, en este caso Scrum.\\

Se utilizarán iteraciones llamadas Sprints que tendrán una duración preestablecida de entre 2 y 4 semanas, logrando con esto obtener una retroalimentación continua y rápida del desarrollo del sistema.\\

Para la organización y planificación de las tareas se utilizará el sistema de Trello con su versión FREE. \\

Para realizar las actividades requeridas para la implementación del sistema se definirán tareas semanalmente apoyándose en el tablero de SCRUM que proporciona el sistema Trello. Estas tareas se definirán y dividirán entre los dos estudiantes encargados de realizar el Trabajo Final de Grado.\\

Durante la implementación del sistema se realizarán tres tipos de pruebas paralelamente para asegurar el correcto funcionamiento del mismo:

\begin{itemize}
	\item Pruebas unitarias.
	\item Pruebas de integración.
	\item Pruebas funcionales.\\
\end{itemize}

El equipo estará formado por los siguientes roles:
\begin{itemize}
	\item Scrum Master: Será rotativo entre los dos estudiantes encargados de realizar el proyecto.
	\item Product Owner: Será la Directora de Tesis Sofia Perez.
	\item Sprint Team: Estará compuesto por los dos estudiantes encargados de realizar el proyecto.\\
\end{itemize}

\section*{\underline{RESUMEN TÉCNICO}}
\begin{figure}[H]
	\centering
	\includegraphics[width=\textwidth]{imagenes/resumen_tecnico.png} 
	\caption{Diagrama de bloques}
\end{figure}

\section*{\underline{PROGRAMACIÓN DE ACTIVIDADES}}
-------- COMPLETAR DIAGRAMA DE GANT ------

\section*{\underline{PROGRAMACIÓN DE RECURSOS}}
Se utilizarán los siguientes recursos y plataformas de desarrollo:
\begin{itemize}
	\item Dos estaciones de trabajo con macOS Big sur, versión 11.2.3.
	\item Entornos de programación de JetBrains con licencia educativa.
	\item Visual Studio Code.
	\item Trello para el seguimiento de las tareas.
	\item Oracle VirtualBox.
	\item Github como versionador de código.
	\item Jenkins como herramienta de Integración Continua y Despliegue Continuo (CI/CD).
	\item Postman free edition.
\end{itemize}

El producto se desplegará sobre un servidor provisto por el Departamento de Computación e Informática, para publicarlo bajo un subdominio de su página web.

\section*{\underline{FACILIDADES REQUERIDAS AL IUA}}
Se hará uso de los equipos e infraestructura disponible en el Departamento de Computación e Informática.


\section*{\underline{PRESUPUESTO}}
\begin{enumerate}[(a)]
	\item Estimación de costos de equipamiento a utilizar en el proyecto:\\
Valor de un servidor físico con características necesarias:

		\begin{itemize}
			\item Memoria RAM 16GB DDR4.
			\item Disco duro SATA de 4TB.
			\item Procesador escalable Intel® Xeon® 3204 (6 núcleos, 1,9 GHz, 85 W).
			\item Fuente de alimentación 550w.
			\item Estado: Disponible y provisto por el Depto. de Computación e Informática de la Facultad de Ingeniería del CRUC IUA. Costo: \$0.-.
		\end{itemize}

	\item Estimación de costos de desarrollo:
		\begin{itemize}
			\item Se estima que el costo de desarrollo para realizar este proyecto es de 3600 USD, siendo el valor en hora de trabajo de 20 USD, con un total de 180 horas estimadas.
		\end{itemize}
\end{enumerate}

\section*{\underline{FUENTES DE FINANCIAMIENTO}}
No corresponde por tratarse de un trabajo final para obtener el título de grado de la carrera.

\section*{\underline{RIESGOS ESPERADOS Y SUPUESTOS ASUMIDOS}}
Los riesos esperados son:
\begin{itemize}
	\item Corte de luz, caída de los dominios del IUA o cualquier otro tipo de incidente que perjudique el dominio en donde estará desplegado el sistema ya que los tesistas no se harán cargo del Hosteo.
	\item Si bien los tesistas no son expertos en el dominio, sí lo es su Directora de tesis.\\
\end{itemize}

Los supuestos asumidos son:
\begin{itemize}
	\item Los equipos provistos por el CRUC IUA estarán disponibles desde la fecha de inicio del proyecto.
	\item El equipo del proyecto UNDEX estará disponible para brindar la información necesaria y requerida en el proyecto, así como realizar pruebas y toda actividad derivada de la metodología de desarrollo empleada.\\
\end{itemize}


\section*{\underline{INVERSIÓN REQUERIDA}}
Los tesistas harán uso de equipos e infraestructura disponible en el Departamento de Computación e Informática y aporte de equipos propios.
\begin{itemize}
	\item Servidores (CRUC IUA).
	\item 2 estaciones de trabajo personales (Estudiantes).\\
\end{itemize}

\section*{\underline{PROYECCIÓN DE COSTOS DE OPERACIÓN Y MANTENIMIENTO}}
- COMPLETAR ----


\section*{\underline{ANÁLISIS DE VIABILIDAD COMERCIAL}}
No corresponde dado que se enmarca en un proyecto UNDEX.


\section*{\underline{ANÁLISIS FINANCIERO}}
No corresponde dado que se trata de un trabajo final de carrera.
%\footnote{https://www.pagina12.com.ar/325647-software-el-pais-buscara-cuadruplicar-los-puestos-de-trabajo}

\section*{\underline{ESTUDIO AMBIENTAL}}
COMPLETAR

\section*{\underline{ESTUDIO SOCIAL}}
La industria tecnológica y de desarrollo de software ha tenido un gran crecimiento en los últimos años. La Argentina no es ajena a este proceso: Actualmente la industria del software posee alrededor de 115.000 empleos y se proyecta para el año 2030 unos 500.000 en total o incluso más \textbf{\cite{EmpleosUltAnios}}.\\

En una industria en pleno crecimiento, la inserción de la mujer sigue presentando desafíos:
\begin{itemize}
	\item Aunque en los últimos 15 años se duplicó la participación de mujeres en la industria de IT, todavía la brecha frente a la participación de hombres en esta misma industria es muy grande (70\% hombres, 30\% mujeres). (colocar referencia)
	\item Solo el 30\% de los estudiantes de carreras de Ciencia, Tecnología, Ingeniería y Matemática (CTIM) que se registran en universidades públicas y privadas son mujeres. \textbf{\cite{MujeresCtim}}\\
\end{itemize}

En adición, la industria de Software es una de las pocas que ha mantenido su crecimiento durante los últimos años y posee uno de los mejores salarios en Argentina.\\

La brecha de género afecta también a la innovación y al crecimiento tanto del campo tecnológico como el desarrollo económico y social de un país. Los programas de inclusión vienen a tratar de disminuir esa brecha de género en tecnología, acercando experiencias formativas a mujeres adolescentes.\\

Es de suma importancia poder tener la posibilidad de realizar un seguimiento de cada alumno que hace estos programas de inclusión para poder sacar conclusiones, métricas y tomar decisiones con el fin de mejorar estas experiencias y por qué no , aumentar la cantidad de mujeres en el rubro de la tecnología. \\

\section*{\underline{EVALUACIÓN ECONÓMICA}}
No corresponde dado que se trata de un trabajo final de carrera.
