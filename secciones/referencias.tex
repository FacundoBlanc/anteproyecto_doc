\begin{thebibliography}{0}
  \bibitem[1]{ResolucionUndex} Resolución 441 de 2019 [Universidad de la Defensa Nacional]. Por la cual se establece el orden de los proyectos declarados como pertinentes y aprobados en la convocatoria 2019 del Programa UNDEX de la secretaría de Extensión. 23 de diciembre de 2019.
  \bibitem[2]{EmpleosUltAnios} Matias, K. (2021). Economía del conocimiento. 13 de agosto de 2021, de Página12 website: https://www.pagina12.com.ar/325647-software-el-pais-buscara-cuadruplicar-los-puestos-de-trabajo
  \bibitem[3]{DuplMujeres} Mujeres en la industria del software. (2020). 11 de agosto de 2021, de cessi Argentina website: https://www.cessi.org.ar/ver-noticias-en-los-ultimos-15-anios-se-duplico-la-participacion-de-las-mujeres-dentro-de-la-industria-del-software-2617
  \bibitem[4]{MujeresCtim} Ana inés, V., \& Cecilia, L. (2019). Un potencial con barreras: la participación de las mujeres en el área de ciencia y tecnología en Argentina. 13 de agosto de 2021, de Banco interamericano de Desarrollo website: https://publications.iadb.org/publications/spanish/document/Un\_potencial\_con\_barreras\_la
  \_participación\_de\_las\_mujeres\_en\_el\_área\_de\_Ciencia\_y\_Tecnología\_en\_Argentina\_es\_es.pdf 
  \bibitem[5]{ChicasEnTecnologia} Chicas en Tecnología. (2021) website: https://chicasentecnologia.org/
  \bibitem[6]{LibroPDS} Jacobson, I., Booch, G., \& Rumbaugh, J. (2000). EL PROCESO UNIFICADO DE DESARROLLO DE SOFTWARE. Addison Wesley.
  \bibitem[7]{LibroSW} Sommerville, I. (2011). Ingeniería de Software (9na ed.). Pearson.
  \bibitem[8]{LibroSOA} Erl,T. (2008). SOA Design Patterns. PRENCTICE HALL.
   
\end{thebibliography}
